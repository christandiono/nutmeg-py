%% LyX 1.6.4.1 created this file.  For more info, see http://www.lyx.org/.
%% Do not edit unless you really know what you are doing.
\documentclass[english]{article}
\usepackage[T1]{fontenc}
\usepackage[latin9]{inputenc}
\usepackage{babel}

\begin{document}

\section{Nutmeg In Python}

Python is an open-source, general purpose, object oriented programming
language that is gaining popularity as a tool for scientific computing.
As an interpreted language with robust object model support, Python
allows a wide variety of programming styles, from line-by-line scripting
to abstracted, reusable library code. Though not specifically developed
for numerics and computation, its strenghts include an emphasis on
legibility and ease-of-use, system portability, and straightforward
access to system libraries. Additionally, there is a very stable,
well developed stack of basic computational tools developed by the
scientific Python community. First among the many commonly used tools
are: Numpy for n-dimensional arrays; Scipy for a wealth of computational
code, much of it being a Python layer over established, validated
libraries such as LAPACK and FFTPACK; and Matplotlib which provides
interactive and scriptable 2D plotting tools that emulate MATLAB plotting.
All these features provide a convenient computing environment for
the development of modern scientific data processing systems whose
scope may expand over time, and whose core functionality typically
demands a design ranging from optimized algorithms to complex data
models for physical phenomena.

Nutmeg has begun a small scale transition from MATLAB to Python. To
date, the membrane from one system to the other lies between voxelwise
source reconstruction, and statistical post processing and visualization. 


\subsection{From MATLAB Data And Routines To Python Objects}

The workflow for a Nutmeg based analysis that incorporates Python
tools presents both a design challenge and a technical data translation
problem. The latter is a solved problem, thanks to code from SciPy
enabling I/O between Numpy arrays and MATLAB data contained in {}``mat''
files. The former allows the use of Python's object model. 

Nutmeg-Py's core includes very simple data models which, abstractly,
have immutable data and metadata, have methods to interrogate or transform
the data in some fashion, and finally can read and write itself on
disk without loss of precision. The Python analog to the MATLAB {}``struct''
containing a time-frequency reconstruction, the TFBeam, is an example
of such an object that is extensively employed in the code. Another
object in this category is the TimeFreqSnPMResults, which represent
the results from a statistical test and can be used to generate thresholds
and maps.

Another common design pattern employed in Nutmeg-Py is the modeling
of a related set of processing task using a hierarchy of classes.
This is done by defining a common parent classes that specifies or
even performs the bulk of the data organization and computation, and
then defining variations of the task with subclasses that prepare
data in task-specific ways and inherit the common functionality. Nutmeg-Py
takes advantage of this pattern in its recreation of Nutmeg's non-parametric
statistical processing machinery. 


\subsection{Visualization}

The project to transition Nutmeg into a Python based toolkit has also
generated a small but powerful visualization effort named Xipy (cross-modality
Imaging in Python), which lies somewhat under the umbrella of the
seminal Nipy (neuroimaging in Python) project. The main ambition of
Xipy is to provide a flexible and extensible system for displaying
brain imagery from various data sources (eg, anatomical MR, statistical
maps, diffusion tracks) in the same 3D scene. The graphics technology
behind most displays in Xipy is Mayavi <\textcompwordmark{}< ref?
>\textcompwordmark{}>, which is a Python visualization codebase which
itself uses VTK <\textcompwordmark{}< ref? >\textcompwordmark{}> for
all of its graphical heavy lifting.

Xipy is in fact totally decoupled from Nutmeg-Py. Visualization of
results from Nutmeg and Nutmeg-Py within Xipy is enabled by a richly
featured plugin contained in the Nutmeg-Py package which implements
stereotyped overlay and thresholding interfaces (see figure X). In
principle, this this overlay plugin architecture could be exploited
for displaying any dataset that can be mapped to a volumetric grid
coregistered to an underlying image. 
\end{document}
